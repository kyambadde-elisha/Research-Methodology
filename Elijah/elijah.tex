\documentclass[a4paper,12pt]{article}
\begin{document}
\begin{center}
\begin{normalsize}

\textbf{{\Huge MAKERERE UNIVERSITY} } \\
\vspace{1cm}
\textbf{{\Large FACULTY OF COMPUTING AND INFORMATICS TECHNOLOGY}} \\
\vspace{1cm}
\textbf{DEPARTMENT OF COMPUTER SCIENCE} \\
\textbf{BACHELOR OF SCIENCE IN COMPUTER SCIENCE} \\
\vspace{1cm}
\textbf{BIT 2207 RESEARCH METHODOLOGY} \\
\textbf{YEAR 2} \\

\vspace{4cm}
\textbf{\underline{{\Large DATA COLLECTION CONCEPT FOR}}\\
\vspace{0.5cm}
\underline{{\Large RESEARCH ON TRAFFIC JAM}}\\
\vspace{0.5cm}
\underline{{\Large HANDLING ON THE ROADS OF}}\\
\vspace{0.5cm}
\underline{{\Large KAMPALA CAPITAL CITY, UGANDA}}\\}

\vspace{3cm}
\textbf{\sc KYAMBADDE ELISHA } \\
\textbf{\sc Reg No: 16/U/6450/PS } \\
\textbf{\sc std No: 216004999}\\
\vfill 
\end{normalsize}
\end{center}
\newpage

 
\section{METHOD ONE: Observation and Informal interview}
With this method I had to visit the different road branches and streets of Kampala at different time intervals namely morning hours, afternoon and evening hours taking snap shots of how dense or less trafficked the different roads were. I did this over a period of one week in order to get a point of consistence in my statistical results and be in position to formulate a thesis for my case.

Under this method I also approached the different traffic officers who are placed along different road stations across the streets of the capital who offered me some information that was very vital for my research. With the period they have been working on these roads they were able to enlighten me more on what times of the day have less or more traffic and how they try to handle the challenge and the different hours of the day.
 
\section{METHOD TWO: GPS and Satellite image analysis}                                                                                                                                                                                    
    
With the aid aid of the GPS tracking system, I was able to visualize the different coordinates showing the traffic distribution on these different streets of the capital,  these coordinates and statistics indicate  that some road branches have heavy traffic in the morning, others in the evening hours and some in the afternoon and these were attributed to different factors ranging from work ,returning home and lunch at favorite spots around the capital among other factors.

I also analyzed different satellite images of the city’s roads taken at different intervals of the day to be able to capture the difference in the traffic on the different road branches of the capital. Now with this visual analysis I was able to get the differences and variations in the traffic distribution around the city.

\end{document}